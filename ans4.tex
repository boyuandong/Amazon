\documentclass{article}
\usepackage[utf8]{inputenc}
\usepackage[english]{babel}
\usepackage{hyperref}
\usepackage{geometry}
 \geometry{
 a4paper,
 total={170mm,257mm},
 left=10mm,
 top=10mm,
 }
 \usepackage{tikz}
 \usepackage{multirow}
\usepackage{xcolor}
\title{\bf{cmput 355 2020  assignment4}}
\author{Boyuan Dong}
\begin{document}
   \maketitle
   \begin{enumerate}
  \item Boyuan \\
  \item Boyuan Dong\\
   \item I coded for the game Amazon and I also write the report.\\
   \item Actually, the code ttt\_classic.py and go\_player.py in class is very helpful for me starting the code.\\
   \item The Game of the Amazons is a board game played by two players.  White moves first, and the players alternate moves thereafter. Each move consists of two parts:
   \begin{enumerate}
   \item First, move one of one's own amazon stone as a queen move in chess (there are eight directions).
   \item Second, after moving, the amazon shoots an arrow from its landing position to another position, using another queenlike move.
   \end{enumerate}
   An arrow, like an amazon, cannot cross or enter a square where another arrow has landed or an amazon of either color stands.\\
   The last player to be able to make a move wins. Draws are impossible.
   Here is the link from Wikipedia:  \url{https://en.wikipedia.org/wiki/Game_of_the_Amazons}\\
   \item Summary:
   \begin{itemize}
   \item My original goal is write a player for game of Amazons, and this player will work properly.
   \item I think I achieve the goal.
   \item I am very proud of that I did a lot of error checks for the input, help instructions will prompt for most of situations. Besides of that, it will also prompt all suggest legal moves when player make an illegal move.
   \item I tried to use numpy to print a grid like go game or ttt game as we seen in class, but failed. Then I just simply use the python instead. That's quite disappointing as I spent a lot of time on this part.\\
   Also, the player can not change to opponent automatically.\\
   The player is not be able to undo moves so far, that's what I need to keep working on.
   \item If I will continue to work on this project later : 
   I'll try to let the player check the opponent, and change to the opponent automatically after each move.\\
   Also, I will find a better way get all legal moves, maybe make a class for Position like go\_player.py did, making the code looks more clean.\\
   I probably will implement the undo method for my player as well.\\
   Then get a resolver for this game.
   \end{itemize}
   \item Compete Amazons player with go\_player.py and ttt\_classic.py: 
   \begin{center}
    \begin{tabular}{| p{4.5cm} | p{2cm} | p{2cm} | p{2cm} | p{5cm} |}
    \hline
    Compete performance&Amazon & Go Player & TTT Classic Player & Summary \\ \hline
    Use numpy& No & Yes & Yes & \\ \hline
    Number of class & Amazon & Position &TransposType\newline Transpos\newline Cell\newline Position & \\ \hline
    Be able to undo & No & Yes & Yes & Maybe next time I will stored the board states to make it be able to undo moves. \\ \hline
    The number of input positions each time & 2 & 1 & 1 & Unlike Go and TTT, Amazon need to move the stones already exist, there are 8 stones in total, need to enter the position of the stone, and the destination position each time want to make a move. \\ \hline
    Be able to genmove\newline (finds value of all moves using alphabeta search) &No & No & Yes & ttt\_classic is also a solver for searching a solution.\\ \hline
    Be able to modify the gameboard size& Yes & Yes & No & Amazon is able to modify the board size, but the size could only be even number.\\ \hline
    Time of showboard function (seconds) & 8e-05  & 2e-04  & 9e-05 & amazon player is the fastest to draw the board.Mainly because I did not use numpy.\\ \hline
    Time of find all possible moves\newline (seconds) &3.5e-05 & 1.5e-04 & 2e-05 & Amazon use the while loops to find all possible queen moves in 8 directions. Instead get all legal moves, go\_player use request- move method to check if the move is legal. \\ \hline
    There are help instructions & Yes & Yes& Yes & Besides the help menu, I also add the suggest moves/coords when player make a illegal move. \\ \hline 
    \end{tabular}
\end{center}
   \item I think it's a player good for players to try and play. I am happy since it works properly. But there are some other updates I can do to make it even greater.\\
\end{enumerate}
\textbf{Working Diary }\\
I choose this project because it's an interesting 2-player board game. And it's a good game for me to practice what I learned in CMPUT355.\\
\begin{itemize}
\item \textbf{November 2, 2020  3 hours: } \\
I search for the board games for this project, and finally picked Game of Amazons.\\
\item \textbf{November 9, 2020  5 hours: }\\
I go over the ttt\_classic.py and go\_player.py and try to use numpy, but failed.\\
Then I started to simply write a player for Amazon by using python.\\
\item \textbf{November 16, 2020 8 hours: }\\
Keep working on Amazon, I finished and updated player\_amazon.py\\
Test and play around with the player.\\
Working on the assignment report, also make a video record for this game.
\end{itemize}
\end{document}